\documentclass[a4paper,welsh,british,twocolumn]{article}
\usepackage{babel}
\usepackage[utf8]{inputenc}
\usepackage[tt=lining]{cfr-lm}
\usepackage{enumitem,geometry,url,fancyref}
\usepackage{csquotes}
   \MakeAutoQuote{‘}{’}
   \MakeAutoQuote*{“}{”}
\geometry{scale=.9}
\setlength{\columnseprule}{0.4pt}
\urlstyle{sf}
\title{\LaTeX{} Package Recommendations}
\author{cfr}
\date{}
\usepackage{fancyhdr}
\fancyhf{}
\renewcommand*\headrulewidth{0pt}
\fancyhf[cf]{%
  Find packages in the Comprehensive \TeX{} Archive Network (CTAN) at \url{ctan.org}.
  Browse by topic at \url{ctan.org/topic}.}
\pagestyle{fancy}
\begin{document}
\pdfinfo{%
  /Title    (LaTeX Package Recommendations)
  /Subject  (LaTeX)
  /Keywords (LaTeX, package)}
\maketitle\thispagestyle{fancy}
\newlist{pkgdescription}{description}{1}
\setlist[pkgdescription]{font=\bfseries\ttfamily}
\newcommand*\lpack[1]{\texttt{\bfseries #1}}
\section{General}
You should almost always use:
\begin{pkgdescription}
  \item[babel] Pass \verb|welsh,british| to your class.
  \item[inputenc] Load with option \verb|utf8|; \verb|\input{ix-utf8enc.dfu}|.
  \item[fontenc] Load with option \verb|T1|.
  \item[textcomp]
  \item[microtype]
\end{pkgdescription}
\section{Document Layout}
If you are using a standard class (e.g.\ \lpack{article}, \lpack{book} or \lpack{report}):
\begin{pkgdescription}
  \item[geometry] to change page dimensions.
  \item[fancyhdr] for custom headers/footers.
  \item[footmisc] for customised footnotes.
  \item[titling] to use document metadata after \verb|\maketitle|.
\end{pkgdescription}
\section{Mathematics}
\begin{pkgdescription}
  \item[mathtools] for enhanced \lpack{amsmath}.
  \item[amssymb] for more symbols, scripts.
  \item[ntheorem] for enhanced theorem environments.
\end{pkgdescription}
\section{Quotes \& Quoting}
\begin{pkgdescription}
  \item[csquotes] for context- and language-sensitive quotations and quotation marks. Recommended if using \lpack{biblatex}.
\end{pkgdescription}
\section{Citations \& Bibliographies}
\begin{pkgdescription}
  \item[biblatex] Load with option \verb|backend=biber|.
\end{pkgdescription}
\section{Cross-Referencing}
\begin{pkgdescription}
  \item[fancyref] for enhanced cross-references.
  \item[cleverref] for enhanced cross-references.
\end{pkgdescription}
\section{Lists}
\begin{pkgdescription}
  \item[enumitem] for custom lists.
  \item[glossaries] for glossaries and lists of acronyms.
\end{pkgdescription}
\section{Tables}
\begin{pkgdescription}
  \item[array] for enhanced tabular environments.
  \item[booktabs] for professional quality tables.
  \item[longtable] for multi-page tables.
  \item[tabularx] for tables with specified width.
  \item[threeparttable] for tables with notes.
  \item[multirow] for cells spanning multiple rows.
\end{pkgdescription}
\section{Floats}
\begin{pkgdescription}
  \item[caption] to customise captions.
  \item[float] more options for floats.
  \item[subcaption] for sub-figures, sub-tables and sub-captions.
  \item[floatrow] for aligned sub-figures.
  \item[rotating] to rotate floats.
\end{pkgdescription}
\section{Hyperlinks}
\begin{pkgdescription}
  \item[hyperref] for hyperlinks.
  \item[bookmark] for enhanced bookmarks.
\end{pkgdescription}
\section{Images \& Colour}
\begin{pkgdescription}
  \item[graphicx] to load external images.
  \item[xcolor] for colour.
\end{pkgdescription}
\section{Diagrams}
\begin{pkgdescription}
  \item[tikz] for diagrams.
  \emph{Many} specialised extensions available.
  \item[pgfplots] for plots.
  Includes \lpack{pgfplotstable} for data tables.
\end{pkgdescription}
\section{External Data}
\begin{pkgdescription}
  \item[datatool] for data manipulation.
  \item[textmerg] for merging text.
\end{pkgdescription}
\section{Version Control}
\begin{pkgdescription}
  \item[svn-multi] for use with \verb|subversion|.
  \item[gitinfo2] for use with \verb|git|.
\end{pkgdescription}
\end{document}
