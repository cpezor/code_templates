%%%%%%%%%%  GENERAL %%%%%%%%
\usepackage[spanish,es-sloppy]{babel}   % Language definition
\usepackage[utf8]{inputenc}             % Support encoding
\usepackage[T1]{fontenc}                % Support font
\usepackage[scaled]{helvet}             % Arial-like font
\renewcommand*{\familydefault}{\sfdefault}

\usepackage{textcomp}                   % With LaTex, but it's not necessary
\usepackage{setspace}                   % Set spacing

%%%%%%%%%%  LAYOUT %%%%%%%%
\usepackage{geometry}   % Page dimensions
 \geometry{
 a4paper,
 left=4 cm,
 right=3 cm,
 top=3 cm,
 bottom=3.5 cm
 }
\usepackage{fancyhdr}                   % Custom headers/footers

%%%%%%%%%%  MATHEMATICS %%%%%%%%
\usepackage{mathtools,amsthm,amssymb}   %  Enhanced amsmath, symbol, theorems

%%%%%%%%%%  QUOTES AND QUOTING %%%%%%%%
\usepackage[autostyle=true]{csquotes}   % Generate language-dependent quotes in the bibliography

%%%%%%%%%%  CITATIONS %%%%%%%%%%%%%%%
% \usepackage[backend=biber,  style=apa]{biblatex}    % Load bibliography
% \addbibresource{<++>.bib}

%%%%%%%%%%  LIST %%%%%%%%%%%%%%%%%%%%
\usepackage{enumitem}                   % Loads enumerate, itemize, description

%%%%%%%%%%  TABLES %%%%%%%%%%%%%%%%%%%%
\usepackage{array}                      % Tabular environment

%%%%%%%%%%  FLOATS %%%%%%%%%%%%%%%%%%%%
\usepackage{float}                      % For option [H], force to put.

\usepackage{subcaption}                 % Subfigures, subtables, subcaption
% \usepackage{floatrow}                 % Align sub-figures
% \usepackage{roating}                  % Rotate floats

%%%%%%%%%%  IMAGES AND COLOUR %%%%%%%%%%%%%%%%%%%%
\usepackage{graphicx}     
\DeclareGraphicsExtensions{.jpg,.jpeg,.png,.eps,.ps,.pdf}
\graphicspath{{imagenes/}}

\usepackage[usenames,dvipsnames,svgnames]{xcolor}                % For colour
\definecolor{granate}{RGB}{113,22,16}
\definecolor{gris}{RGB}{154,153,157}
\definecolor{arena}{RGB}{230,217,170}
\definecolor{azul}{rgb}{0.03, 0.15, 0.4}
\definecolor{negro}{rgb}{0, 0, 0}

%%%%%%%%%%  HYPERLINKS %%%%%%%%%%%%%%%%%%%%
\usepackage[final]{hyperref}  % para agregar hiper vínculos en el PDF generado
\hypersetup{
	colorlinks=true,            % false: boxed links; true: colored links
	linkcolor=azul,             % color of internal links
	citecolor=azul,             % color of links to bibliography
	filecolor=magenta,          % color of file links
	urlcolor=azul         
}

\usepackage[noabbrev]{cleveref}

%%---------------------------------------

\fancypagestyle{plain}{
\fancyhf{}
\renewcommand{\headrulewidth}{0.5 pt}
\renewcommand{\footrulewidth}{0.5 pt}


\lhead[]{\textit{\scriptsize UNIVERSIDAD NACIONAL DE INGENIERÍA \\ \scriptsize {FACULTAD DE CIENCIAS}}}
\rhead[]{\textit{\scriptsize \leftmark}}


% \lfoot[]{\textit{\scriptsize {\titulo} \\ Bach. {\autor}}}
\lfoot[]{\textit{\scriptsize {\titulo}} \\}
\rfoot[]{\small \thepage}
}
%----------------------------------------------


%%%%% TOC AND TIT
\usepackage{titlesec}       % Titulos de SECCIONES
\usepackage{tocloft}        % Titulos de ÍNDICES

\setcounter{secnumdepth}{3} % Para que ponga 1.1.1.1 en subsubsecciones

% Centrado del título del ÍNDICE / LISTA DE FIGURAS / LISTA DE CUADROS

\renewcommand{\cfttoctitlefont}{\hfill \normalfont\normalsize\bfseries}
\renewcommand{\cftaftertoctitle}{\hfill}

\renewcommand{\cftlottitlefont}{\hfill\normalfont\normalsize\bfseries}
\renewcommand{\cftafterlottitle}{\hfill}

\renewcommand{\cftloftitlefont}{\hfill\normalfont\normalsize\bfseries}
\renewcommand{\cftafterloftitle}{\hfill}


% Formato de los CAPÍTULOS, SECCIONES Y SUBSECCIONES
\titleformat{\chapter}[block]{\normalfont\normalsize\bfseries}{CAPÍTULO \thechapter:}{0.5em}{\normalsize}
\titlespacing*{\chapter}{0pt}{-10 pt}{5pt}

\titleformat{\section}[block]{\normalfont\normalsize}{\thesection}{0.5em}{\normalsize}
\titlespacing*{\section}{0pt}{10 pt}{5 pt}

\titleformat{\subsection}[block]{\normalfont\normalsize}{\thesubsection}{0.5em}{\normalsize}
\titlespacing*{\subsection}{0pt}{10 pt}{5 pt}

\titleformat{\subsubsection}[block]{\normalfont\normalsize}{\thesubsubsection}{0.5em}{\normalsize}
\titlespacing*{\subsubsection}{0pt}{10 pt}{5 pt}

% Posicionamiento vertical de TOC, LOT and LOF

\setlength{\cftbeforelottitleskip}{1pt} 
\renewcommand{\cftafterlottitleskip}{12pt}

\setlength{\cftbeforeloftitleskip}{1pt} 
\renewcommand{\cftafterloftitleskip}{12pt}

\setlength{\cftbeforetoctitleskip}{16pt} 
\renewcommand{\cftaftertoctitleskip}{12 pt}


% Cambiando las etiquetas de las FIGURAS y TABLAS (Caption y autoref)
\addto\captionsspanish{\renewcommand{\figurename}{\footnotesize FIGURA N°}}
\addto\extrasspanish{\def\figureautorefname{ Figura N°}}
 
\addto\captionsspanish{\renewcommand{\tablename}{\footnotesize TABLA N°}}
\addto\extrasspanish{\def\tableautorefname{ Tabla N°}} 


% Espaciamiento dentro del índice

\setlength{\cftbeforechapskip}{2mm}
\renewcommand\cftchapafterpnum{\vskip6pt}
\renewcommand\cftsecafterpnum{\vskip5pt}
\renewcommand\cftsubsecafterpnum{\vskip5pt}


%Agregar la palabra CAPITULO al TOC, FIGURA a LOF y TABLA al LOT

\renewcommand{\cftchappresnum}{CAPÍTULO }
\renewcommand{\cftchapaftersnum}{:}
\renewcommand{\cftchapnumwidth}{7em}

\renewcommand{\cftfigpresnum}{Figura N° }
%\renewcommand{\cftfigaftersnum}{:}
\renewcommand{\cftfignumwidth}{6.85 em}

\renewcommand{\cfttabpresnum}{Tabla N° }
%\renewcommand{\cftfigaftersnum}{:}
\renewcommand{\cfttabnumwidth}{6.5 em}


%%%%%%%%%%  DEFINE FOR THE AUTOR %%%%%%%%%%%%%%%%%%%%

%%%%% TOC
\renewcommand{\thechapter}{\Roman{chapter}}
\renewcommand{\theequation}{\arabic{chapter}.\arabic{equation}} 
\renewcommand{\thesection}{\arabic{chapter}.\arabic{section}}  
\renewcommand{\thetable}{\arabic{chapter}.\arabic{table}}  
\renewcommand{\thefigure}{\arabic{chapter}.\arabic{figure}}

\newcommand\N{\ensuremath{\mathbb{N}}}
\newcommand\R{\ensuremath{\mathbb{R}}}
\newcommand\Z{\ensuremath{\mathbb{Z}}}
\renewcommand\O{\ensuremath{\emptyset}}
\newcommand\Q{\ensuremath{\mathbb{Q}}}
\newcommand\C{\ensuremath{\mathbb{C}}}

% Define shortcuts
\let\implies\Rightarrow
\let\impliedby\Leftarrow
\let\iff\Leftrightarrow
\let\epsilon\varepsilon

%%%%% THEOREMS
\DeclareMathOperator{\sen}{sen}

\newtheorem{obs}{Observación}[section]

\theoremstyle{definition}
% Aqui defino entronos de teoremas y demostraciones
\newtheorem{defi}{Definicion}[section]
\newtheorem{prop}{Proposicion}[section]
\newtheorem{teo}{Teorema}[section]
\newtheorem{col}{Corolario}[section]
\newtheorem{lema}{Lema}[section]
