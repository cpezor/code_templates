% Plantilla para la preparación de los artículo para JITEL 2019

\newcommand{\CLASSINPUTinnersidemargin}{18mm}
\newcommand{\CLASSINPUToutersidemargin}{12mm}
\newcommand{\CLASSINPUTtoptextmargin}{20mm}
\newcommand{\CLASSINPUTbottomtextmargin}{25mm}

\documentclass[10pt,conference, twoside,a4paper]{IEEEtran}

%%%%%%%%%%%%%%%%%%%%%%%%%%%%%%%%%%%%%%%%%%%%%%%%%%%%%%%%%%%%%%%%%%%%%%%%%%%%%%%%%%%%%%%%%%%%%%%%%%%%%%%%%%%%%%%%%%%%%%%%%%%%%%%%%%%%%%%%%%%%%%%%%%%%%%%%%%

% PAQUETES

% Para que salgan las tildes y demás caracteres en castellano...
% o bien
\usepackage[T1]{fontenc}     
\usepackage[utf8]{inputenc}
\usepackage[spanish,es-sloppy]{babel}     

% Paquetes matematicos
\usepackage{mathtools,amsthm,amssymb}  


% Fuente Times...
\usepackage{times}
% \usepackage{kantlipsum} %No se para que sirve

% Figuras en formato .png, .ps, pdf o eps
\usepackage{here}     
\usepackage{graphicx}
\usepackage{subfigure}
\DeclareGraphicsExtensions{.png,.eps,.ps,.pdf}

% Definimos carpeta de imagenes
\graphicspath{{imagenes/}}

% Bibliografia (biber)
% \usepackage[backend=biber, style=authoryear-icomp]{biblatex}    
% \addbibresource{<++>}

\usepackage{fancyhdr}

% Formato y tipografía de URL, direcciones de correo...
\usepackage{url}

%Page margins
\usepackage[left=2cm,top=2.5cm,right=2cm, bottom=2.5cm]{geometry}

\usepackage[usenames]{color}

% Defino el seno
\DeclareMathOperator{\sen}{sen}

% Aqui defino entronos de teoremas y demostraciones
\newtheorem{obs}{Observación}[section]


\fancyhf{}

% Encabezado paginas pares. 
% Escribir el apellido de los autores, separados por comas: Apellido1, Apellido2,Apellido3, 2019.
\fancyhead[CE]{\makebox(0,0)[c]{\it Pezo, Calle, Peñaloza, 2020.}}


% Encabezado paginas impares. 
% Escribir el título del articulo aqui.
\fancyhead[CO]{\makebox(0,0)[c]{\it Nombre del experimento}}


\renewcommand{\headrulewidth}{0pt}
\renewcommand{\footrulewidth}{0pt}
\pagestyle{fancy}


\usepackage{eso-pic}

%%%%%%%%%%%%%%%%%%%%%%%%%%%%%%%%%%%%%%%%%%%%%%%%%%%%%%%%%%%%%%%%%%%%%%%%%%%%%%%%%%%%%%%%%%%%%%%%%%%%%%%%%%%%%%%%%%%%%%%%%%%%%%%%%%%%%%%%%%%%%%%%%%%%%%%%%%%%

% Comienza el documento

\begin{document}
\AddToShipoutPictureBG*{%
  \AtPageUpperLeft{%
    \setlength\unitlength{1in}%
 	\hspace{2cm}%% 
 	 	\makebox(0,-2)[l]{%Encabezado página principal
			\begin{tabular}{l r} 
			\multicolumn{1}{p{12cm}}{\vspace{-1.2cm}\includegraphics[scale=0.50]{UNI-logo.png}} & 						\multicolumn{1}{p{4cm}}{\raggedleft\small\usefont{T1}{phv}{m}{it} Laboratorio de circuitos de la Facultad de Ciencias\\ (UNI 2020),\\ Lima (Perú), \\ \today} \tabularnewline 
			\end{tabular}
 }
}}

% Título del artículo
\title{\vspace{3cm}Experimento N°X \\ \vspace{1mm} Nombre del experimento} %Cambiar el numero

% Datos de los autores
\author{\authorblockN{Pezo Roel, Cesar, Calle Huaman, Alerson, Peñaloza Acevedo, David}
\authorblockA{Curso CF3E1 A \\ Facultad de Ciencias,\\
Universidad Nacional de Ingeniería\\
% Dirección Postal.\\
\url{cpezor@uni.pe}, \url{alerson.calle.h@uni.pe}, \url{dapeñalozaa@uni.pe}}
}


\maketitle

\begin{abstract}
  <++>
\end{abstract}

\begin{keywords}
% escribe aqui entre 5 y 7 palabras claves.
  <++>
\end{keywords}


% Fundamento Teórico
\section{\uppercase{Introducción}}

<++>

\section{\uppercase{Objetivos}}

\subsection{Generales}

<++>

\subsection{Específicos}

<++>

\section{\uppercase{Datos}}

<++>



\section{\uppercase{Cálculos, gráficos y resultados}}

<++>


\section{\uppercase{Discusión de resultados}}

<++>


\section{\uppercase{Conclusiones}}

<++>


\section{\uppercase{Observaciones}}

<++>


\section{\uppercase{Anexos}}

<++>


% Bibliografia
% \printbibliography %  Funciona con  biber


% Ejemplos
\subsection{Figuras y tablas}


\begin{table}[H]
\centering
\caption{Tabla de ejemplo.}
\label{tab:tabla}
\begin{tabular}{c c c c}
\hline
\hline
 Protocolo & Escenario 1 & Escenario 2 & Escenario 3 \\
\hline
 $P_1$ & $0.1$ & $0.3$ & $0.2$ \\
 $P_2$ & $0.2$ & $0.3$ & $0.5$ \\
 $P_4$ & $0.2$ & $0.1$ & $0.2$ \\
 $P_5$ & $0.3$ & $0.3$ & $0.5$ \\
\hline
\hline
\end{tabular}
\end{table}

\begin{figure}[H]
\centerline{
\includegraphics[width=5.5cm]{UNI-logo.png}
}
\caption{Imagen de la UNI}
\label{fig:protocols}
\end{figure}


\end{document}
