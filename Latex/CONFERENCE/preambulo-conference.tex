% PREAMBULO PARA CONFERENCE

%%%%%%%%%%  GENERAL %%%%%%%%
\usepackage[spanish,es-sloppy]{babel}   % Language definition
\usepackage[utf8]{inputenc}             % Support encoding
\usepackage[T1]{fontenc}                % Support encoding
% \usepackage{times}
\usepackage[scaled]{helvet}             % Arial-like font
\renewcommand*{\familydefault}{\sfdefault}

\usepackage{textcomp}                   % With LaTex, but it's not necessary

%%%%%%%%%%  MATHEMATICS %%%%%%%%
\usepackage{mathtools,amsthm,amssymb}  %  Enhanced amsmath, symbol, theorems

%%%%%%%%%%  QUOTES AND QUOTING %%%%%%%%
\usepackage[autostyle=true]{csquotes} % Generate language-dependent quotes in the bibliography

%%%%%%%%%%  CITATIONS %%%%%%%%%%%%%%%
% \usepackage[backend=biber,  style=apa]{biblatex}    % Load bibliography
% \addbibresource{<++>.bib}

%%%%%%%%%%  LIST %%%%%%%%%%%%%%%%%%%%
\usepackage{enumitem}                 % Loads enumerate, itemize, description

%%%%%%%%%%  TABLES %%%%%%%%%%%%%%%%%%%%
\usepackage{array}                 % Tabular environment

%%%%%%%%%%  FLOATS %%%%%%%%%%%%%%%%%%%%
\usepackage{float}                % For option [H], force to put.
\usepackage{subcaption}           % Subfigures, subtables, subcaption

%%%%%%%%%%  IMAGES AND COLOUR %%%%%%%%%%%%%%%%%%%%
\usepackage{graphicx}     
\DeclareGraphicsExtensions{.jpg,.jpeg,.png,.eps,.ps,.pdf}
\graphicspath{{imagenes/}}

\usepackage{xcolor}                % For colour

%%%%%%%%%%  DEFINE FOR THE AUTOR %%%%%%%%%%%%%%%%%%%%

%%%%% COLOURS
\definecolor{granate}{RGB}{113,22,16}
\definecolor{gris}{RGB}{154,153,157}
\definecolor{arena}{RGB}{230,217,170}
\definecolor{azul}{rgb}{0.03, 0.15, 0.4}
\definecolor{negro}{rgb}{0, 0, 0}

%%%%%%%%%%  HYPERLINKS %%%%%%%%%%%%%%%%%%%%
\usepackage[final]{hyperref} % para agregar hiper vínculos en el PDF generado
\hypersetup{
	colorlinks=true,       % false: boxed links; true: colored links
	linkcolor=azul,        % color of internal links
	citecolor=azul,        % color of links to bibliography
	filecolor=magenta,     % color of file links
	urlcolor=azul         
}

\newcommand\N{\ensuremath{\mathbb{N}}}
\newcommand\R{\ensuremath{\mathbb{R}}}
\newcommand\Z{\ensuremath{\mathbb{Z}}}
\renewcommand\O{\ensuremath{\emptyset}}
\newcommand\Q{\ensuremath{\mathbb{Q}}}
\newcommand\C{\ensuremath{\mathbb{C}}}

% Define shortcuts
\let\implies\Rightarrow
\let\impliedby\Leftarrow
\let\iff\Leftrightarrow
\let\epsilon\varepsilon

%%%%% THEOREMS
\DeclareMathOperator{\sen}{sen}

\newtheorem{obs}{Observación}[section]

\theoremstyle{definition}
% Aqui defino entronos de teoremas y demostraciones
\newtheorem{defi}{Definicion}[section]
\newtheorem{prop}{Proposicion}[section]
\newtheorem{teo}{Teorema}[section]
\newtheorem{col}{Corolario}[section]
\newtheorem{lema}{Lema}[section]

%%%%% TITLE
\title{\titulo}

\author{\IEEEauthorblockN{\autor}
\IEEEauthorblockA{  Curso IF471 B - Sistemas Digitales \\ Facultad de Ciencias,\\
Universidad Nacional de Ingeniería\\
\url{cpezor@uni.pe}}
}
